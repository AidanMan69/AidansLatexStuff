\documentclass{jhwhw}
\author{Aidan Garcia}
\usepackage[utf8]{inputenc}
\usepackage{mathrsfs}
\usepackage{graphicx}
\usepackage{amsmath}
\usepackage{amsfonts}
\usepackage{amssymb}
\usepackage{spalign}
\usepackage{nicefrac}
\usepackage{tabto}
\usepackage{parskip}
\title{8.2: Nonhomogeneous Linear Systems}

\begin{document}

\maketitle

\section{Linear Systems of ODEs with Distinct Eigenvalues}

{\Large\underline{\textbf{Example 1}}}
\\[\baselineskip]
Solve the following system of ODEs if \(x_1(0)=0\) and \(x_2(0)=-4\).
\\
\begin{align*}
x'_1 & = x_1 + 2x_2 \\
x'_2 & = 3x_1 + 2x_2
\end{align*} 

\begin{align*}
\vec{x'} = \mathbb{A}\vec{x} \quad \textrm{where} \quad \mathbb{A} = \begin{pmatrix}
    1 & 2 \\ 3 & 2
\end{pmatrix}
\end{align*}
\\
To get the eigenvalues of the system, you must find the characteristic equation. You must use the forumula \(\det(\mathbb{A} - \lambda \mathbb{I}  ) = 0\).
\\
\begin{align*}
\det(\mathbb{A} - \lambda \mathbb{I} ) & = 0 \\
\det \begin{pmatrix}
    1-\lambda & 2 \\ 3 & 2-\lambda 
\end{pmatrix} & = 0 \\
(1-\lambda)(2-\lambda)-(3)(2) & = 0 \\
2-\lambda - 2 \lambda + \lambda ^2 - 6 & = 0 \\
\lambda ^2 - 3\lambda  - 4 & = 0 \\
(\lambda +1)(\lambda - 4) & = 0 \\
\therefore \lambda_1 = -1 \quad and \quad & \lambda_2 = 4 
\end{align*}

To get the eigenvectors with the eigenvalues, plug the eigenvalues into the equation \((\mathbb{A}- \lambda \mathbb{I})\vec\eta = 0\). We will start off by solving for \(\lambda = -1\).
\\
\begin{align*}
&(\mathbb{A}-\lambda \mathbb{I})\vec\eta_1 = 0\\ \\
    &\begin{pmatrix}
    1-\lambda & 2 \\ 3 & 2-\lambda
\end{pmatrix}\begin{pmatrix}
    \eta_1 \\ \eta_2
\end{pmatrix} = 0 \\ \\
    \text{Now plug in } & \text{the eigenvalue } \lambda = -1. \\ \\
    &\begin{pmatrix}
        2 & 2 \\ 3 & 3
    \end{pmatrix}\begin{pmatrix}
        \eta_1 & \eta_2
    \end{pmatrix} = 0 \\ \\
    \text{Use an augmented matrix to } &\text{find } \eta_1 \text{ and } \eta_2 \text{ and do row operations}. \\ \\ 
    &\spalignaugmat{2 2 0; 3 3 0} \text{ } -\frac{3}{2}R_1 + R_2\rightarrow R_2 \\ \\ 
    &\spalignaugmat{2 2 0; 0 0 0} \text{ } \frac{1}{2}R_1 \rightarrow R_1 \\ \\
    &\spalignaugmat{1 1 0; 0 0 0} \text{ } \\ \\ 
    &\eta_1 + \eta_2 = 0 \\
    &\eta_1 = -\eta_2 \\ \\
    &\text{Now build } \vec\eta_1 \\ \\
    \vec\eta_1 = \begin{pmatrix}
        \eta_1 \\ \eta_2
    \end{pmatrix} &= \begin{pmatrix}
        -\eta_2 \\ \eta_2
    \end{pmatrix} = \begin{pmatrix}
        -1 \\ 1
    \end{pmatrix}\eta_2 \\ \\
    \therefore \vec\eta_1 = \begin{pmatrix}
        -1 \\ 1
    \end{pmatrix} &\text{ for } \lambda = -1 \\
\end{align*}

To get the second eigenvector, use the same formula \((\mathbb{A} - \lambda\mathbb{I})\vec\eta_2 = 0\). You must use the second eigenvalue \(\lambda = 4\) to get the second eigenvector. Since we know that using the formula results in an augmented matrix, we can just plug in the second eigenvalue right away. 
\\ \\ \\
When \(\lambda = 4\), it results in this augmented matrix:
\\
\begin{align*}
    \spalignaugmat{ -3 2 0; 3 -2 0} \text{ } R_1 + R_2 \rightarrow R_2 \\ \\
    \spalignaugmat{-3 2 0; 0 0 0} \\ \\
    -3\eta_1 + 2\eta_2 = 0 \\
    \eta_1 = \frac{2}{3}\eta_2 \\
\end{align*}

Now we build \(\vec\eta_2\).

\begin{align*}
    \vec\eta_2=\begin{pmatrix}
        \eta_1 \\ \eta_2
    \end{pmatrix} = \begin{pmatrix}
        \frac{2}{3}\eta_2 \\ \eta_2
    \end{pmatrix} = \begin{pmatrix}
        \frac{2}{3} \\ 1
    \end{pmatrix}\eta_2 = \frac{1}{3}\begin{pmatrix}sd
        2 \\ 3
    \end{pmatrix}\eta_2 \\ \\
    \therefore \vec\eta_2 = \begin{pmatrix}
        2 \\ 3
    \end{pmatrix} \text{ for } \lambda = 4
\end{align*}
\\
Since we have both eigenvalues and both eigenvalues, we can build the general solution to the sytem of ODEs. We can plug all of the values we have obtained into the general form of the general solution \(\vec x(t) = C_1 e^{\lambda{t}} \eta_1 + C_2 e^{\lambda{t}}\eta_2\).
\\ \\
Therefore the general solution to the system is:
\\
\begin{align*}
    \boxed{\vec x(t) = C_1 e^{-t} \begin{pmatrix} -1 \\ 1 \end{pmatrix} + C_2 e^{4t} \begin{pmatrix} 2 \\ 3 \end{pmatrix}}
\end{align*}
\\ 
We have the general solution, however since this is an initial value problem we must plug in the given values. It is stated that \(x_1(0) = 0\) and \(x_2(0) = -4\) which shows us that \(\vec x(0) = \begin{pmatrix} 0 \\ -4 \end{pmatrix}\) because \(\vec x(t) = \begin{pmatrix} x_1 \\ x_2 \end{pmatrix}\) therefore \(\vec x(0) = \begin{pmatrix} x_1(0)\\ x_2(0)\end{pmatrix}\).
\\ \\ \\ \\ \\ \\ \\ \\ \\ \\ \\
Now we plug in the initial conditions and solve for \(C_1\) and \(C_2\):
\\
\begin{align*}
    \vec x(t) = C_1 e^{-t} \begin{pmatrix} -1 \\ 1 \end{pmatrix} + C_2 e^{4t} \begin{pmatrix} 2 \\ 3 \end{pmatrix} \\ \\
    \vec x(0) = C_1 \begin{pmatrix} -1 \\ 1 \end{pmatrix} + C_2 \begin{pmatrix} 2 \\ 3 \end{pmatrix} \\ \\
    \begin{pmatrix} 0 \\ -4 \end{pmatrix} = C_1 \begin{pmatrix} -1 \\ 1 \end{pmatrix} + \begin{pmatrix} 2 \\ 3 \end{pmatrix} \\ \\
    \begin{pmatrix} 0 \\ -4 \end{pmatrix} = \begin{pmatrix} -C_1 \\ C_1 \end{pmatrix} + \begin{pmatrix} 2C_2 \\ 3C_2 \end{pmatrix} \\ \\
    \begin{pmatrix} 0 \\ -4 \end{pmatrix} = \begin{pmatrix} -C_1 + 2C_2 \\ C_1 + 3C_2 \end{pmatrix} \\ \\
    \begin{pmatrix} 0 \\ -4 \end{pmatrix} = \begin{pmatrix} -1 & 2 \\ 1 & 3 \end{pmatrix} \begin{pmatrix} C_1 \\ C_2 \end{pmatrix} \\ \\
    -\frac{1}{5}\begin{pmatrix} 3 & -2 \\ -1 & -1 \end{pmatrix} \begin{pmatrix} 0 \\ -4 \end{pmatrix} = \begin{pmatrix} C_1 \\ C_2 \end{pmatrix} \\ \\ 
    \begin{pmatrix} -\nicefrac{8}{5} \\ -\nicefrac{4}{5} \end{pmatrix} = \begin{pmatrix} C_1 \\ C_2 \end{pmatrix} \\ \\
    \therefore C_1 = -\frac{8}{5} \text{ and } C_2 = -\frac{4}{5}
\end{align*}
\\
Since we have found both \(C_1\) and \(C_2\), we can plug them into our general solution.
\\ \\ 
Therefore the specific solution to the system given the initial conditions \(x_1(0) = 0\) and \(x_2(0) = -4\) is:
\\
\begin{align*}
    \boxed{\vec x(t) = -\frac{8}{5} e^{-t} \begin{pmatrix} -1 \\ 1 \end{pmatrix} + -\frac{4}{5} e^{4t} \begin{pmatrix} 2 \\ 3 \end{pmatrix}}
\end{align*}
\\
\end{document}
