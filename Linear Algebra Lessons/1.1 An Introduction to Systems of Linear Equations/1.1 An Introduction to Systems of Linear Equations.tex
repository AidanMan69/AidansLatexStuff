\documentclass{jhwhw}
\author{Aidan Garcia}
\usepackage[utf8]{inputenc}
\usepackage{mathrsfs}
\usepackage{graphicx}
\usepackage{amsmath}
\usepackage{amsfonts}
\usepackage{amssymb}
\usepackage{spalign}
\usepackage{nicefrac}
\usepackage{tabto}
\usepackage{parskip}
\usepackage{systeme}
\title{1.1: An Introduction to Systems of Linear Equations}

\begin{document}

A linear equation in \(n>1\) variables \(x_1,x_2,x_3,\ldots,x_n\) has the form
\\
(\(\ast\)) \(a_1 x_1 + a_2 x_2 + a_3 x_3 +\ldots+a_n x_n=b\)
\\

here \(a_i\) is a constant for any \(1 \leq i \leq n\) and \(b\) is a constant
\\

\(a_i\) are the coefficients of (\(\ast\)) \\
\(x_i\) are the variables \\
\(a_1\) is the leading coefficient \\
\(x_1\) is the leading variable
\\

\textbf{Example: Linear Equations}
\\
a) \(2x-3y=1\) (2 variables) \\
\(\rightarrow\) represents a line of the xy plane 
\\
b) \(x+y-z=2\) (3 variables) \\
\(\rightarrow\) represents a plane in the xyz coordinate system
\\ \\

\textbf{Example: Non-Linear Equations}
\\
1) \( \boxed{x^2} + y + z =0\) \(\leftarrow\) non-linear function \(x\)
\\
2) \(\boxed{xy} + z = -1\) \(\leftarrow\) x and y are not independent of eachother
\\
3) \(\boxed{\cos x} + y + z = 3\) \(\leftarrow\) non-linear function of \(x\)
\\ \\

\underline{Definition (Solution and Solution Set of (\(\ast\))):}
\\

A solution of (\(\ast\)) is a sequence of n real numbers  \(s_1, s_2, s_3,\ldots,s_n\) that satisfy  (\(\ast\))
\\

The set of solutions of (\(\ast\)) is called a solution set
\\

Remark: The solution of (\(\ast\)) is a linear equation in \(n\) variables does have a parametric representation.
\\ \\

\textbf{Example: Find a parametric representation of each linear equation}
\\

a) \(\boxed{1}\) \(2x_1 - 3x_2 = 1\) (a linear equation in \(n=2\) represents a line in \(\mathbb{R} ^2\))
\\
let \(x_2=t\), here \(t \in \mathbb{R}\)
\\

\begin{align*} 
\text{then } \boxed{1} &\Leftrightarrow 2x_1 - 3t = 1 \\
&\Leftrightarrow 2x_1 = 1 + 3t \\
&\Leftrightarrow x_1 = \frac{1}{2} + \frac{3}{2}t \\
\end{align*}

Then the solution set of \(\boxed{1}\) is 
\\
\begin{align*} 
    &\left\{(x_1 , x_2) | x_1 = \frac{1}{2} + \frac{3}{2}t \text{ and } x_2 = t \text{, } t \in \mathbb{R} \right\}\\
    &\boxed{= \left\{\left(\frac{1}{2} + \frac{3}{2}t, t\right) | t \in \mathbb{R} \right\}} \text{ A paramaterization of the solution set}\end{align*}
\\ \\

b) \(\boxed{2}\) \(x_1 + 2x_2 - x_3 = 4\)\\ \\
Let \(x_3=t, \, t \in \mathbb{R}\) is a parameter \\
Let \(x_2 = s, \, s \in \mathbb{R}\) is a parameter
\\ \\
Then \\
\begin{align*}  
    &\Leftrightarrow x_2 + 2s - t = 4 \\
    &\Leftrightarrow x_2 = -2s + t + 4, \, t \in \mathbb{R} \text{ and } s \in \mathbb{R}
\end{align*}
\\ \\
Then the solution set of \(\boxed{2}\) is
\begin{align*}
&\left\{(x_1, x_2, x_3) | \, x_2 = -2s +t +4, \,x_2 =s, \,x_3 =t, \,s \in \mathbb{R} \text{ and } t \in \mathbb{R} \right\}\\
&\boxed{= \left\{(-2s+t+4,\, s,\, t) | \,s \in \mathbb{R} \text{ and }t \in \mathbb{R} \right\}}
\end{align*}
\\ \\

\underline{Definition (System of linear equations in \(n \geq 1\) variables):}
\\

A system of \(m \geq 1\) linear equation in \(n \geq 1\) variables is a set of \(m\) equations, each of which is linear in the same \(n\) variables
\\

(\(\ast\)) \begin{align*} 
a_{11} x_1 + a_{12} x_2 + a_{13} x_3 + \ldots + a_{1n} x_n &= b_1\\
a_{21} x_1 + a_{22} x_2 + a_{23} x_3 + \ldots + a_{2n} x_n &= b_2\\
a_{31} x_1 + a_{32} x_2 + a_{33} x_3 + \ldots + a_{3n} x_n = b_3\\
\vdots\\
a_{m1} x_1 + a_{m2} x_2 + a_{m3} x_3 + \ldots + a_{mn} x_n &= b_m
\end{align*}
\\

A solution of (\(\ast\)) is a sequence of numbers \(s_1, s_2, s_3, \ldots, s_n\) that is a solution of each equation of the system.
\\

Remark (On the number of solutions of a system (\(\ast\)) of linear equations):
\\

For a given system of linear equations (\(\ast\)) Precisely one of the following statements is true:
\\

1) The system (\(\ast\)) has exactly one solution (system is consistent) \\
2) The system (\(\ast\)) has infinitely many solutions (system is consistent) \\
3) The system (\(\ast\)) has no solutions (system is inconsistent) \\
\\

\textbf{Examples: Different Cases for Systems of Linear Equations}
\\

a) \begin{align*}
    (\ast) \quad \begin{cases}
        x+y=0 & \quad \textcircled{1} \\ x-y=1 & \quad \textcircled{2}
    \end{cases}
\end{align*}

(\(\ast\)) is a system of \(m =2 \) linear equations in \(n=2\) variables \\
(\(\ast\)) has one solution (\(\frac{1}{2}, -\frac{1}{2}\)). The system is consitent.
\\ \\

b)  \begin{align*} 
    (\ast) \quad \begin{cases} 
    x+y=1 & \quad \textcircled{1} \\ 2x+2y=2 & \quad \textcircled{2}  
\end{cases}
\end{align*}

(\(\ast\)) is a system of \(m=2\) linear equations and \(n=2\) variables \\
observe that \(2=2\) \\
then (\(\ast\)) \(\Leftrightarrow\) \(x+y=1\) hence the solution is
\begin{align*} 
\left\{(x,y) \,|\, x+y=1 \right\} &= \left\{ (x,y) \, | \, y=1-x \right\} \\
&= \left\{(x, \, 1-x) | \, x \in \mathbb{R} \right\}
\end{align*}
\\ \\
c) \begin{align*} 
(\ast) \quad \begin{cases} x+y=-1 & \quad \textcircled{1} \\ x+y=1 & \quad \textcircled{2} \end{cases}
\end{align*}
\\
Note that \(\textcircled{1}\) and \(\textcircled{2}\) imply that: \(1=-1\) which is impossible, then (\(\ast\)) has no solutions. The system is inconsistent.
\\ \\

\underline{Definition (system in a row-echelon form):}
\\ 

A system of linear equations is in row-echelon form when it has a stair-step pattern with leading coefficients of 1. To solve is we use a back substitution.
\\

\textbf{Examples: Row-Echelon Form}
\\

a) \begin{align*} 
(\ast) \quad \systeme{x-2y+3z=9,y+3z=5,z=2} \quad \quad \begin{aligned} &\textcircled{1} \\ &\textcircled{2} \\ &\textcircled{3} \end{aligned}
\end{align*}
\\
(\(\ast\)) is in row-echelon form, it has a stair-step pattern and the leading coefficients are 1.
\\ \\

b) \begin{align*} 
(\ast) \quad \systeme{x+2y=1,3y=5} \quad \quad \begin{aligned} &\textcircled{1} \\ &\textcircled{2}\end{aligned}
\end{align*}
\\
(\(\ast\)) has a stair-step pattern, but the leading coefficient of the second row isn't 1. Therefore, (\(\ast\)) is not in row-echelon form.
\\ \\

c) \begin{align*} 
(\ast) \quad \systeme{x-y+z=1,y-z=0,y+6z=3} \quad \quad \begin{aligned} &\textcircled{1} \\ &\textcircled{2} \\ &\textcircled{3} \end{aligned}
\end{align*}
\\
The leading coefficients of (\(\ast\)) are 1. (\(\ast\)) does not have a stair-step pattern, therefore (\(\ast\)) is not in row-echelon form.
\\ \\

\underline{Definition (equivalence of systems of linear equations:)}
\\

Two systems of linear equations are \underline{equivalent} when they have the same solution set (the systems must have the same number of linear equations and the same variables).
\\

\textbf{Example: System Equivalency}
\\
\begin{align*} 
&(\ast) \quad \boxed{\begin{cases} 2x+2y=2 \\ x-y=1 \end{cases}} \quad \textcircled{1}\\
&\Leftrightarrow \quad \boxed{\begin{cases} x+y=1 \\ x-y=1 \end{cases}} \quad \textcircled{2}\\
&\Leftrightarrow \quad \boxed{\begin{cases} x=1 \\ y=0 \end{cases}} \quad \textcircled{3}
\end{align*}
\\
The solution set is \(\left\{(1,0)\right\}\) hence the systems \(\textcircled{1}\), \(\textcircled{2}\), and \(\textcircled{3}\) are equivalent as they have the same solution set.
\\

Remark: To solve a system that is not in a row-echelon form we write it as an \underline{equivalent} \underline{system} in row-echelon form.
\\

\textbf{Question}: What are the possible operations that we can apply to the equations of a system (\(\ast\)) to obtain an equivalent system?
\\
\textbf{Answer}: The following are operations that we can apply to the equations in (\(\ast\)) to obtain an equivalent system:\\
1) Interchanging the rows \\
2) Multiplying an equation by a non-zero constant \\
3) Adding a multiply of an equation to another equation
\\

 Remark: Rewriting a system of linear equations in a row-echelon form involves a chain of equivalent systems using operations 1), 2), and 3): the process is called a Gaussian Eelimination.
 \\

\textbf{Examples: Gaussian Elimination}
\\

a) \begin{align*} 
    (\ast) \quad \systeme{x-2y+3z=9, -x+3y=-4, 2x-5y+5z=17} \quad \quad \begin{aligned} &\textcircled{1} \\ &\textcircled{2} \\ &\textcircled{3}\end{aligned}
\end{align*}
\\
(\(\ast\)) is a system of \(m=3\) linear equations in \(n=3\) variables.
\\
To solve (\(\ast\)) we apply a Gaussian Elimination:
\\

\begin{align*} 
(\ast) \, &\Leftrightarrow \systeme{x-2y+3z=9, -x+3y=-4, -y-z=-1} \quad \quad \begin{aligned} &\textcircled{1} \\ &\textcircled{2} \\ &(-2)\textcircled{1}+\textcircled{3} \rightarrow \textcircled{3}'\end{aligned}
\\ \\
&\Leftrightarrow \quad \systeme{x-2y+3z=9, y+3z=5, -y-z=-1} \quad \quad \begin{aligned} &\textcircled{1} \\ &\textcircled{1}+\textcircled{2} \rightarrow \textcircled{2}' \\ &\textcircled{3}'\end{aligned}
\\ \\
&\Leftrightarrow \quad \systeme{x-2y+3z=9, y+3z=5, 2z=4} \quad \quad \begin{aligned} &\textcircled{1} \\ &\textcircled{2}' \\ &\textcircled{2}'+\textcircled{3}' \rightarrow \textcircled{3}''\end{aligned}
\\ \\
&\Leftrightarrow \quad \systeme{x-2y+3z=9, y+3z=5, z=2} \quad \quad \begin{aligned} &\textcircled{1} \\ &\textcircled{2}' \\ &\frac{1}{2}\textcircled{3}'' \rightarrow \textcircled{3}'''\end{aligned}
\end{align*}
\\

The system is in a row-echelon form, to solve it we use a back substitution, substitution \(\boxed{z=2}\) in \(\textcircled{2}'\) yields:
\\

\begin{align*} 
y+3(2)=5\\
\Leftrightarrow \boxed{y=-1}
\end{align*}
\\

substituting \(y=-1\) and \(z=2\) in \(\textcircled{1}\) yields:
\\

\begin{align*} 
x-2(-1)+3(2)=9\\
\Leftrightarrow x+2+6=9\\
\implies \boxed{x=1}
\end{align*}
\\
The system is consistent, it has a unique solution and the solution set is \(\left\{(1,-1,2)\right\}\)
\\ \\

b) \begin{align*} 
(\ast) \quad \systeme{x-3y+z=1, 2x-y-2z=2, x+2y-3z=-1} \quad \quad \begin{aligned} &\textcircled{1} \\ &\textcircled{2} \\ &\textcircled{3}\end{aligned}
\end{align*}
\\
(\(\ast\)) is a system of \(m=3\) linear equations in \(n=3\) variables. 
\\
To solve (\(\ast\)) we apply a Gaussian Elimination:
\\

\begin{align*} 
(\ast) \, &\Leftrightarrow \quad \systeme{x-3y+z=1, 2x-y-2z=2, 5y-4z=-2} \quad \quad \begin{aligned} &\textcircled{1} \\ &\textcircled{2} \\ &(-1)\textcircled{1}+\textcircled{3} \rightarrow \textcircled{3}''\end{aligned}
\\ \\ 
&\Leftrightarrow \quad \systeme{x-3y+z=1, 5y-4z=0, 5y-4z=-2} \quad \quad \begin{aligned} &\textcircled{1} \\ &(-2)\textcircled{1}+\textcircled{2} \rightarrow \textcircled{2}' \\ &\textcircled{3}''\end{aligned}
\end{align*}
\\
The system is inconsistent and has no solutions because of the equations \(5y-4z=0\) and \(5y-4z=-2\). They are the same but equal different values.
\\ \\

c) \begin{align*} 
(\ast) \quad \systeme{y-z=0, x-3z=-1, -x+3y=1} \quad \quad \begin{aligned} &\textcircled{1} \\ &\textcircled{2} \\ &\textcircled{3}\end{aligned}
\end{align*}
\\
(\(\ast\)) is a system of \(m=3\) linear equations in \(n=3\) variables. 
\\
To solve (\(\ast\)) we apply a Gaussian Elimination:
\\

\begin{align*} 
(\ast)\,&\Leftrightarrow \quad \systeme{x-3z=-1, y-z=0 , -x+3y=1} \quad \quad \begin{aligned} &\textcircled{2} \\ &\textcircled{1} \\ &\textcircled{3}\end{aligned}
\\ \\
&\Leftrightarrow \quad \systeme{x-3z=-1, y-z=0 , 3y-3z=0} \quad \quad \begin{aligned} &\textcircled{2} \\ &\textcircled{1} \\ &\textcircled{2}+\textcircled{3} \rightarrow \textcircled{3}'\end{aligned}
\\ \\
&\Leftrightarrow \quad \systeme{x-3z=-1, y-z=0} \quad \quad \begin{aligned} &\textcircled{2} \\ &\textcircled{1}\end{aligned}
\end{align*}
\\
Let \(z=t\)\\
\(\textcircled{1} \implies y = t\)\\
\(\textcircled{2} \implies x=-1 +3t\)\\
then \begin{align*} 
\begin{cases} x=-1+3t \\ y=t \\ z=t \end{cases} \text{, }\, t \in \mathbb{R} 
\end{align*}
\\
Hence (\(\ast\)) is a consistent system, it has infinitely many solutions. The solution set (on a paramaterized form) is:
\\
\begin{align*} 
    &\left\{(x,\,y,\,z) |\, x=-1+3t,\,y=t,\,z=t,\,t \in \mathbb{R}\right\}\\
    &=\left\{(-1+3t,\,t,\,t) |\,t \in \mathbb{R}  \right\}
\end{align*}
\\
This represents a line in space.
\\ \\

\end{document}