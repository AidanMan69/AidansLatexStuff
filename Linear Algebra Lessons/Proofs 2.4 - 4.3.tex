\documentclass{jhwhw}
\author{Aidan Garcia}
\usepackage[utf8]{inputenc}
\usepackage{mathrsfs}
\usepackage{graphicx}
\usepackage{amsmath}
\usepackage{amsfonts}
\usepackage{amssymb}
\usepackage{amsthm}
\usepackage{spalign}
\usepackage{nicefrac}
\usepackage{tabto}
\usepackage{parskip}
\usepackage{systeme}
\usepackage[english]{babel}
\newtheorem{theorem}{Theorem}

\theoremstyle{definition}
\newtheorem{definition}{Definition}

\theoremstyle{remark}
\newtheorem*{remark}{Remark}

\theoremstyle{example}
\newtheorem*{example}{Example}
 
\title{Proofs 2.4 - 4.3 (from notes)}

\begin{document}
\section{Elementary Matrices}
44 in 2.4\\
1) Prove that \(A\) is idempotent if and only if \(A^T\) is idempotent\\
2) Prove that if \(A\) is an \(n \times n\) matrix that is idempotent and invertible, then \(A = I_n\)\\
3) Prove that if \(A\) and \(B\) are idempotent and \(AB=BA\), then \(AB\) is idempotent\\
4) Prove that if \(A\) is row-equivalent to \(B\) and \(B\) is row-equivalent to \(C\), then \(A\) is row-equivalent to \(C\)\\
5) Prove that if matrix \(A\) is row equivalent to matrix \(B\) then matrix \(B\) is row equivalent to matrix \(A\)\\
6) Let \(A\) be a nonsingular matrix. Prove that if \(B\) is row-equivalent to \(A\), then \(B\) is also non-singular. 
\section{The Determinant of a Matrix}
\begin{theorem}[Determinant of an Upper or Lower Triangular Matrix]  If \(A\) is triangular (upper or lower matrix of order \(n \geq 1\), and if \(A = [a_{ij}]\) for \(1 \leq n\), \(1 \leq j \leq n\))
    then its determinant is the product of the entries on the main diagonal that is: 
    \begin{align*} \det(A) = |A| &= a_{11} a_{22} a_{33} \ldots a_{nn}\\
    &= \prod_{i=1}^{n} a_{ii} \end{align*} \end{theorem}
\section{Determinants and Elementary Row Oerations}
\begin{theorem}[Elementary Row Operations and Determinants] Let \(A\) and \(B\) be square matrices of the same size, then\\
    1) When \(B\) is obtained from \(A\) by interchanging two rows of \(A\), then \(\det(A) = - \det(B)\)\\
    2) When \(B\) is obtained from \(A\) by adding a multiple of a row of \(A\) to another row, then \(\det(B) = \det(A)\)\\
    3) When \(B\) is obtained from \(A\) by multiplying a row of \(A\) by a non-zero contstant \(c\), then \(\det(B) = c \det(A)\) \end{theorem}
\begin{theorem}[Conditions That Yield a Zero Determinant] If \(A\) is a square matrix and any of the conditions given below is true, then \(\det(A) = 0\)\\
1) An entire row (or an entire column) consists of zeros\\
2) Two rows (or two columns) are equal\\
3) One row (or one column) is a multiple of another row (is a multiple of another column) \end{theorem}
\section{Properties of Determinants}
\begin{theorem}[Determinant of a Finite Matrix Product] If \(A\) and \(B\) are square matrices of order \(n \geq 1\) then: \(\det(AB) = \det(A) \det(B)\) \end{theorem}
\begin{remark} If \(A_1 A_2 A_3 \ldots A_k (k \geq 1) \)\\
If these are matrices of the same order \(n \geq 1\), then
\begin{align*} & \det (A_1 A_2 A_3 \ldots A_k) = \det (A_1) \det (A_2) \det (A_3) \ldots \det(A_k)\\
 & \Leftrightarrow \det (\prod_{i=1}^{n} A_i) = \prod_{i=1}^{n} \det(A_i) \end{align*}\end{remark}
\begin{theorem}[Determinant of a Scalar Multiple of a Matrix] If \(A\) is a square matrix of order \(n \geq 1\) and \(c\) is a scalar, then
\begin{align*} \det(cA) = c^n \det(A) \end{align*} \end{theorem}
1) Let \(A\) and \(B\) be \(n \times n\) matrices such that \(AB = I\). Prove that \(|A| \neq 0 \) and \(B \neq 0\)\\
2) Let \(A\) and \(B\) be \(n \times n\) matrices such that \(AB\) is singular. Prove that either \(A\) or \(B\) is singular\\
3) Let \(A\) be an \(n \times n\) matrix in which the entires of each row sum to zero. Find \(|A|\)\\
4) Prove that the determinant of an invertible matrix \(A\) is equal to \(\pm 1\) when all of the entries of \(A\) and \(A^{-1}\) are integers\\
5) If \(A\) is a square matrix, then \(\det (A) = \det(A^T)\)\\
6) A square matrix is skew-symmetric when \(A^T = -A\). Prove that if \(A\) is an \(n \times n\) skew-symmetric matrix, then \(|A| = (-1)^n |A|\)\\
7) Let \(A\) be a skew-symmetric matrix of odd order. Use the result of exercise 69 (6) to prove that \(|A| = 0\)\\
8) Prove that the \(n \times n\) identity matrix is orthogonal\\
9) Prove that if \(A\) is an orthogonal matrix, then \(|A| = \pm 1\)\\
10) If \(A\) is an idempotent matrix \((A^2 = A)\), then prove that the determinant of \(A\) is either \(0 \text{ or } 1\)\\
11) Let \(S\) be an \(n \times n\) singular matrix. Prove that for any \(n \times n\) matrix B, the matrix \(SB\) is also singular
\end{document}