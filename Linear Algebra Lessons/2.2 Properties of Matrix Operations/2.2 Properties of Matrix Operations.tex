\documentclass{jhwhw}
\author{Aidan Garcia}
\usepackage[utf8]{inputenc}
\usepackage{mathrsfs}
\usepackage{graphicx}
\usepackage{amsmath}
\usepackage{amsfonts}
\usepackage{amssymb}
\usepackage{amsthm}
\usepackage{spalign}
\usepackage{nicefrac}
\usepackage{tabto}
\usepackage{parskip}
\usepackage{systeme}
\usepackage[english]{babel}
\newtheorem{theorem}{Theorem}

\theoremstyle{definition}
\newtheorem{definition}{Definition}

\theoremstyle{remark}
\newtheorem*{remark}{Remark}

\theoremstyle{example}
\newtheorem*{example}{Example}

\title{2.2: Properties of Matrix Operations}

\begin{document}
I) The Algebra of Matrices\\

\begin{theorem}[Properties of matrix addition and scalar multiplication] Let \(A\), \(B\) and \(C\) be \(m*n\) matricies and \(c\), \(d\) be scalars \end{theorem}

1) \(A+B=B+A\) (Commutative property of matrix addition)\\
2) \(A+(B+C)=(A+B)+C\) (Associative property of matrix addition)\\
3) \((cd)A = c(dA)\) (Associative property of scalar multilcation of matricies)\\
4) \(c(A+B)=cA + cB\) (Distributive property of scalar multiplication of matricies)\\
5) \((c+d)A = cA+dA\) (Distributive property of scalar multiplication of matricies)
\\ \\
\underline{Proofs of 1), 2), and 5):}
\\
\begin{proof} Let \(A=[a_{ij}]\) and \(B=[b_{ij}]\) for \(1 \leq i \leq m\), \(1 \leq j \leq n\)\\
then \begin{align*} 
A+B &= [a_{ij} + b_{ij}]\\
&=[b_{ij} + a_{ij}]\\
&=B+A
\end{align*}
\end{proof}
\begin{proof} Let \(A=[a_{ij}]\), \(B=[b_{ij}]\) and \(C=[c_{ij}]\) for \(1 \leq i \leq m\), \(1 \leq j \leq n\)\\
    then \begin{align*} 
    A+(B+C) &= [a_{ij}] + [b_{ij} + c_{ij}]\\
    &= [a_{ij} + b_{ij} + c_{ij}]\\
    &= [(a_{ij} + b_{ij}) + c_{ij}] \, \text{ since the operation of addition is associative in } \mathbb{R}\\
    &= (A+B)+C
    \end{align*}
\end{proof}
\begin{proof} 
    Let \(A=[a_{ij}]\) and \(B=[b_{ij}]\) for \(1 \leq i \leq m\), \(1 \leq j \leq n\) and let \(c \in \mathbb{R}\) be a scalar\\
    then \begin{align*} 
    c(A+B) &= c[a_{ij} + b_{ij}]\\
    &= [c(a_{ij} + b_{ij})]\\
    &= [c a_{ij} + c b_{ij}] \, \text{ by the distributive property of multilication with respect to addition in } \mathbb{R}\\
    &=[c a_{ij}] + [c b_{ij}]\\
    &=c[a_{ij}] + c[b_{ij}]\\
    &=cA + cB
    \end{align*}
\end{proof}
\begin{definition}[Zero matrix] The \(m*n\) matrix denoted by \(0_{mn}\) whose entries are all 0's is called the zero matrix of size \(m*n\) \end{definition}
\begin{align*} 0_{mn} = \begin{bmatrix} 0 & 0 & \ldots & \ldots & 0 \\ 0 & 0 & \ldots & \ldots & 0 \\ 0 & 0 & \ldots & \ldots & 0 \\ \vdots & \vdots & \vdots & \ddots & \vdots \\ 0 & \ldots & \ldots & \ldots & 0  \end{bmatrix} \end{align*}

Note: \(0_{mn} = a_{ij}\) where \(a_{ij}=0\) for any \(1 \leq i \leq m, \, 1 \leq j \leq m\)
\begin{theorem}[Properties of the zero matrix] If \(A\) is an \(m*n\) matrix and \(c \in \mathbb{R}\) is a scalar, then \end{theorem}

1) \(A + 0_{mn} = A\)\\
2) \(A+(-A) = 0_{mn}\)\\
3) If \(cA = 0_{mn}\) then \(c = 0\) or \(A = 0_{mn}\)
\\

\underline{Proof of 1):}
\\
\begin{proof} 
Let \(A=[a_{ij}]\) for \(1 \leq i \leq m\), \(1 \leq j \leq n\)\\
then \begin{align*} 
& \Leftrightarrow \quad c[a_{ij}] = [0]\\
& \Leftrightarrow \quad [c a_{ij}] = [0]\\
& \Leftrightarrow \quad c a_{ij} = 0\\
\end{align*}
\(c=0\) or \(a_{ij} = 0_{mn}\)  for all \( 1 \leq i \leq m\), \(1 \leq j \leq n\)\\
\(\implies \, A = 0_{mn}\)
\end{proof}
\textbf{Example.}
    Let \(A = \begin{bmatrix} -2 & -1 \\ 1 & 0 \\ 3 & -4 \end{bmatrix}\) and \(B = \begin{bmatrix} 0 & 3 \\ 2 & 0 \\ -4 & -1 \end{bmatrix}\) and \(x\) be a \(3*2\) matrix. Solve the matrix equation \(3x+2A=B\) (\(\ast\))\\

let \(x=\begin{bmatrix} x_{11} & x_{12} \\ x_{21} & x_{22} \\ x_{31} & x_{32} \end{bmatrix}\) where \(x_{ij} \in \mathbb{R}\) for \(1 \leq i \leq m\) and \(1 \leq j \leq n\)\\
then \begin{align*} 
(\ast) \, &\Leftrightarrow 3x + 2A = B\\
& \Leftrightarrow 3x + 0_{mn} = B - 2A\\
& \Leftrightarrow 3x = B - 2A\\
& \Leftrightarrow \frac{1}{3}(3x) = (B-2A)\frac{1}{3}\\
& \Leftrightarrow (\frac{1}{3} \centerdot 3)x = \frac{1}{3}(B-2A)\\
& \Leftrightarrow x = \frac{1}{3}(B - 2A) \\
& \Leftrightarrow x = \frac{1}{3}(\begin{bmatrix} 0 & 3 \\ 2 & 0 \\ -4 & -1\end{bmatrix} - 2 \begin{bmatrix} -2 & -1 \\ 1 & 0 \\ 3 & -4 \end{bmatrix})\\
& \Leftrightarrow x = \frac{1}{3}(\begin{bmatrix} 0 & 3 \\ 2 & 0 \\ -4 & -1\end{bmatrix} + \begin{bmatrix} 4 & 2 \\ -2 & 0 \\ -6 & 8 \end{bmatrix})\\
& \Leftrightarrow x = \frac{1}{3}(\begin{bmatrix} 4 & 5 \\ 0 & 0 \\ -10 & 7\end{bmatrix})\\
& \Leftrightarrow x = \begin{bmatrix} \nicefrac{4}{3} & \nicefrac{5}{3} \\ 0 & 0 \\ -\nicefrac{10}{3} & \nicefrac{7}{3} \end{bmatrix}
\end{align*}
\\ \\

\begin{theorem}[Properties of matrix multilication] Let \(A\), \(B\) and \(C\), be matricies (with sizes such that the matrix products given below are defined) and let \(\alpha \in \mathbb{R}\)  be a scalar\end{theorem}

1) \(A(BC) = (AB)C\) (Associative property of multiplication)\\
2) \(A(B+C) = AB+AC\) (Distributive property of matrix mult. w.r.t matrix addition)\\ 
3) \((A+B)C = AC + BC\) (Distributive property of matrix mult. w.r.t matrix addition)\\
4) \(\alpha (AB) = (\alpha A)B = A(\alpha B)\)
\\

\underline{Proof of 2):}
\\
\begin{proof} 
Let \(A=[a_{ij}]\) be an \(m*n\) matrix\\ and let \(B = [b_{ij}]\) and  \(C = [c_{ij}]\) be \(n*p\) matricies, \\
then \(B+C = [b_{ij} + c_{ij}]\) is an \(n*p\) matrix\\
hence \(A(B+C)\) is a well defined matrix of \(m*p\)\\
let \(A(B+C) = [d_{ij}]\)

Recall: If \(A = [a_{ij}]\) is an \(m*n\) matrix, and \(B = [b_{ij}]\) be an \(n*p\) matrix, then \(AB\) is an \(m*p\) matrix and \(AB = [c_{ij}]\) where \(cij = \sum_{k=1}^{n} a_{ik} b_{kj}\)

here, \begin{align*} 
d_{ij} &= \sum_{k=1}^{n} a_{ik}(b_{kj}+c_{kj})\\
&= \sum_{k=1}^{n} a_{ik} b_{kj} + a_{ik} c_{kj}\\
&= \sum_{k=1}^{n} a_{ik} b_{kj} + \sum_{k=1}^{n} a_{ik} c_{kj}\\
&= q_{ij} + r_{ij}
\end{align*}
where \(AB=[q_{ij}] = [\sum_{k=1}^{n} a_{ik} b_{kj}]\)\\
and \(AC=[r_{ij}] = [\sum_{k=1}^{n} a_{ik} c_{kj}]\)\\
hence \(A(B+C)= AB + AC\)
\end{proof}

\begin{remark} Let \(A\) and \(B\) be matricies, with sizes such that products given below are well defined \end{remark}
1) \((A+B)^2 \neq A^2 + 2AB + B^2 \) \(\,\) in general since \((A+B)^2 = (A+B)(A+B) = A^2 +AB + BA + B^2\). However, you must be careful since \(AB \neq BA\) in general.

2) \((A-B)^2 \neq A^2 - 2AB + B^2 \) \(\,\) in general since \((A+B)^2 = (A+B)(A+B) = A^2 -AB - BA + B^2\)
\\ \\

\begin{definition}[Diagonal of a matrix] Let \(A = [a_{ij}]\) be an \(n*n\) matrix, then its \underline{diagonal} consists of the entries \(a_{ii}\) for any \(1 \leq i \leq n\) \end{definition}
\begin{align*} A = \begin{bmatrix} a_{11} & a_{12} & a_{13} & \ldots & a_{1n} \\ a_{21} & a_{22} & a_{23} & \ldots & a_{2n} \\ \vdots & \vdots & \vdots & \ddots & \vdots \\  a_{n1} & a_{n2} & a_{n3} & \ldots & a_{nn} \end{bmatrix} \end{align*}
and its trace is the sum of all diagonal entries, that is: 
\begin{align*} 
Trace(A) &= a_{11} + a_{22} + a_{33} + \ldots + a_{nn}\\
&= \sum_{i=1}^{n} a_{ii}
\end{align*}
\begin{definition}[Identity Matrix] An \(n*n\) matrix, \(n \geq 1\) that has 1 on all its diagonal entries and 0's elsewhere is called the indentity matrix of order n (size \(n*n\)). It is denoted by \(I_n\) \end{definition}
\begin{align*} I_n = [a_{ij}] \text{ where } a_{ij} = \begin{cases} 1 \text{ if } i=j \\ 0 \text{ if } i \neq j \end{cases} \text{where } 1 \leq i \leq n, \, 1 \leq j \leq n\end{align*}
\begin{align*} I_n = \begin{bmatrix} 1 & 0 & 0 & \ldots & 0 \\ 0 & 1 & 0 & \ldots & 0 \\ 0 & 0 & 1 & \ldots & 0 \\ \vdots & \vdots & \vdots & \ddots & \vdots \\ 0 & \ldots & \ldots & \ldots & 1  \end{bmatrix} \end{align*}

\begin{example}
\(I_1 = \begin{bmatrix} 1 \end{bmatrix}, \, I_2 = \begin{bmatrix} 1 & 0 \\ 0 & 1 \end{bmatrix}, \, I_3 = \begin{bmatrix} 1 & 0 & 0 \\ 0 & 1 & 0 \\ 0 & 0 & 1 \end{bmatrix}\)    
\end{example}
\begin{remark} The matrix \(I_n\) (\(n \geq 1\)) serves as the identity element for matrix multiplication \end{remark}

\begin{theorem}[Properties of the identity matrix] If \(A\) is a matrix of size \(m*n\), then \end{theorem}
\begin{align*} & 1) \, AI = A\\
& 2) \, IA = A
\end{align*}

\textbf{Example.}
Consider the \(4*4\) matrix \(A = \begin{bmatrix} 0 & 1 & 0 & 0 \\ 0 & 0 & 1 & 0 \\ 0 & 0 & 0 & 1 \\ 0 & 0 & 0 & 0 \end{bmatrix}\) . Show that there exists a positive integer \(n\) such that \(A^n = 0_{44}\)
\begin{align*} A^2 = AA &= \begin{bmatrix} 0 & 1 & 0 & 0 \\ 0 & 0 & 1 & 0 \\ 0 & 0 & 0 & 1 \\ 0 & 0 & 0 & 0  \end{bmatrix} \begin{bmatrix} 0 & 1 & 0 & 0 \\ 0 & 0 & 1 & 0 \\ 0 & 0 & 0 & 1 \\ 0 & 0 & 0 & 0  \end{bmatrix}\\ \\
&= \begin{bmatrix} 0 & 0 & 1 & 0 \\ 0 & 0 & 0 & 1 \\ 0 & 0 & 0 & 0 \\ 0 & 0 & 0 & 0  \end{bmatrix}
\end{align*}
\begin{align*} 
    A^3 = A^2 A &= \begin{bmatrix} 0 & 0 & 1 & 0 \\ 0 & 0 & 0 & 1 \\ 0 & 0 & 0 & 0 \\ 0 & 0 & 0 & 0  \end{bmatrix} \begin{bmatrix} 0 & 0 & 1 & 0 \\ 0 & 0 & 0 & 1 \\ 0 & 0 & 0 & 0 \\ 0 & 0 & 0 & 0  \end{bmatrix}\\ \\
&= \begin{bmatrix} 0 & 0 & 0 & 1 \\ 0 & 0 & 0 & 0 \\ 0 & 0 & 0 & 1 \\ 0 & 0 & 0 & 0  \end{bmatrix}
\end{align*}
\begin{align*} 
    A^4 = A^3 A &= \begin{bmatrix} 0 & 0 & 0 & 1 \\ 0 & 0 & 0 & 0 \\ 0 & 0 & 0 & 1 \\ 0 & 0 & 0 & 0 \end{bmatrix} \begin{bmatrix}  0 & 0 & 0 & 1 \\ 0 & 0 & 0 & 0 \\ 0 & 0 & 0 & 1 \\ 0 & 0 & 0 & 0  \end{bmatrix}\\ \\
    &= \begin{bmatrix}  0 & 0 & 0 & 0 \\ 0 & 0 & 0 & 0 \\ 0 & 0 & 0 & 0\\ 0 & 0 & 0 & 0\end{bmatrix} \\ \\
&= 0_{44}
\end{align*}
\(\boxed{\therefore n = 4}\)
\\ \\ \\

\textbf{Exercises:}\\
Prove 1), 2), 3) for properties of zero matrix\\
Prove 1), 3), 4) for properties of matrix multiplication



\end{document}


