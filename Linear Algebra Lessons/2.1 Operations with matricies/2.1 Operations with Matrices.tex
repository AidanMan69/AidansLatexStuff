\documentclass{jhwhw}
\author{Aidan Garcia}
\usepackage[utf8]{inputenc}
\usepackage{mathrsfs}
\usepackage{graphicx}
\usepackage{amsmath}
\usepackage{amsfonts}
\usepackage{amssymb}
\usepackage{spalign}
\usepackage{nicefrac}
\usepackage{tabto}
\usepackage{parskip}
\usepackage{systeme}
\title{2.1: Operations with Matrices}

\begin{document}
To represent a matrix, we use one of these conventions:\\
1) an uppercase latter: A, B, C, D...\\
2) A representative element \([a_{ij}], [b_{ij}], [c_{ij}]\)\\ \\
3) A rectangular array: \(\begin{bmatrix} a_{11} & a_{12} & a_{13} & \ldots & a_{1n} \\ a_{21} & a_{22} & a_{32} & \ldots & a_{2n} \\ a_{31} & a_{32} & a_{33} & \ldots & a_{3n} \\ \vdots & \vdots & \vdots & \ddots & \vdots \\ a_{m1} & a_{m2} & a_{m3} & \ldots & a_{mn} \end{bmatrix}\)\\

Remark:\\
1) The matrices that we will consider have real valued entries: \(a_{ij} \in \mathbb{R} \\ \text{ for } 1 \leq i \leq m, \, 1 \leq j \leq n\)\\
2) Matrices, depending on the applications, can have complex-valued entries \(a_{ij} \in \mathbb{C}\)
\\ \\

\underline{Definition: Equality of Matrices}\\
Two matricies \(A = [b_{ij}]\) and \(B = [b_{ij}]\) are equal \(\iff\)\\
1) \(A\) and \(B\) have the same size\\
2) \(a_{ij} = b_{ij} \text{ for any } 1 \leq i \leq m, \, 1 \leq j \leq n\)
\\ \\

Example:
\\

\(A = \begin{bmatrix} -1 & 2 \\ 0 & 3 \end{bmatrix}\), \(\begin{bmatrix} 1 & 4 \\ -1 & 2 \\ 3 & 5 \end{bmatrix} = B\)\\

\(A\) and \(B\) have different sizes, hence are \underline{not} equal\\

\underline{Definitions: Row Matrix, Column Matrix}\\
A matrix that has one column is called a column matrix (or a column vector)\\

Example (Column Matrix):\\

\(A = \begin{bmatrix} 2 \\ -1 \\ 3\end{bmatrix}\)
\\

A matrix that has one row is called a row matrix (or a row vector)\\

Example (Row Matrix):\\

\(B = \begin{bmatrix}  4 & 1 & 3 & 0 \end{bmatrix}\)
\\ \\

\underline{Definition: Matrix Addition}\\
If \(A = [a_{ij}]\) and \(B = [b_{ij}]\) are matricies of sized \(m*n\), then \underline{their sum} is the \(m*n\) matrix.\\

\(A+B = [a_{ij} + b_{ij}] \text{ for } 1 \leq i \leq m, \, 1 \leq j \leq n\)\\

Remark: The sim of two matrices of different sizes is undefined
\\

Example:
\\
a) Let \(A = \begin{bmatrix} -1 & 2 & 4 \\ 0 & 3 & -6 \end{bmatrix}\)\\ \\
and \(B = \begin{bmatrix} 0 & -2 & -4 \\ 1 & 0 & -1 \end{bmatrix}\)\\ \\
Since \(A\) and \(B\) have equal size, then their sum is the \(2*3\) matrix\\ \\
\begin{align*} A + B &= \begin{bmatrix} -1 & 2 & 4 \\ 0 & 3 & -6 \end{bmatrix} + \begin{bmatrix} 0 & -2 & -4 \\ 1 & 0 & -1 \end{bmatrix}\\ \\
&= \begin{bmatrix} -1 + 0 & 2+(-2) & 4+4 \\ 0+1 & 3+0 & -6+(-1) \end{bmatrix}\\ \\
&= \begin{bmatrix} -1 & 0 & 8 \\ 1 & 3 & -7 \end{bmatrix}
\end{align*}
\\

b) Let \(A = \begin{bmatrix}  0 & -1 & 2  \\ 1 & 0 & 4 \\ -1 & 2 & 5 \end{bmatrix}\) \\ \\
and \(B = \begin{bmatrix}  -1 & 0 \\  2 & 1 \\ 3 & 4 \end{bmatrix}\)\\

\(\rightarrow\) the matrices have different sizes hence their sum is undefined\\

Example:\\
Consider the matrices \(A\) and \(B\) given in part a) on the previous examles, then it is easy to verify that: \(A+B = B+A = \begin{bmatrix} -1 & 0 & 8 \\  1 & 3 & -7 \end{bmatrix}\)\\

\(\rightarrow\) In general, if \(A\) and \(B\) are matricies of size \(m*n\), then \(A+B=B+A\), hence the oeration of matrix addition is commutative.\\ \\

\underline{Definition: Matrix Scalar Multilication}\\
If \(A = [a_{ij}]\) is a matrix of size \(m*n\) and \(c \in \mathbb{R}\) is a scalar, then that scalar multiple of \(A\) is the \(m*n\) matrix\\

\(cA = c[a_{ij}] = [c a_{ij}] \text{ for } 1 \leq i \leq m, \, 1 \leq j \leq n\)
\\

Example:
\\
Let \(A = \begin{bmatrix} -1 & 3 & 0 \\ 0 & 1 & 1 \\ 2 & 4 & 0 \end{bmatrix}\) and \(B = \begin{bmatrix} 0 & -4 & 1 \\ 4 & -2 & 0 \\ 1 & 1 & 3 \end{bmatrix}\)\\
then\\

\begin{align*} -2A + 3B &= -2 \begin{bmatrix} -1 & 3 & 0 \\ 0 & 1 & 1 \\ 2 & 4 & 0 \end{bmatrix} + 3 \begin{bmatrix} 0 & -4 & 1 \\ 4 & -2 & 0 \\ 1 & 1 & 3 \end{bmatrix}\\ \\
&= \begin{bmatrix} 2 & -6 & 0 \\ 0 & -2 & -2 \\ -4 & -8 & 0 \end{bmatrix} + \begin{bmatrix}  0 & -12 & 3 \\ 12 & -6 & 0 \\ 3 & 3 & 9 \end{bmatrix}\\ \\
&= \begin{bmatrix} 2 & -18 & 3 \\ 12 & -8 & -2 \\ -1 & -5 & 9 \end{bmatrix}
\end{align*}
\\ \\

\underline{Definition: Matrix Multiplication}\\
If \(A = [a_{ij}]\) is an \(m*n\) matrix\\
and \(B = [b_{ij}]\) is an \(n*p\) matrix\\

Then their product is the matrix \(AB\) of size \(m*p\)\\
here \(AB = [C_{ij}]\), where\\
\begin{align*} c_{ij} &= \sum_{k=1}^{n} a_{ik} b_{kj} \\
&= a_{i1}b_{1j} + a_{i2}b_{2j} + a_{i3}b_{3j} + \ldots + a_{in}b_{nj}
\end{align*}
\\
here \(1 \leq i \leq m, \, 1 \leq j \leq p\)\\

Remark: On multilying a row by a column\\

\(\begin{bmatrix} x & y & z \end{bmatrix} \begin{bmatrix} a \\ b \\ c \end{bmatrix} = \begin{bmatrix} xa & yb & zc \end{bmatrix}\)\\

Example:\\

Let \(A = \begin{bmatrix}  1 & 0 \\  -2 & 4 \\ 3 & 5 \end{bmatrix}\)\\
and \(B = \begin{bmatrix} -2 & 3 \\ -4 & 1 \end{bmatrix}\)\\
then \\

\begin{align*} AB &= \begin{bmatrix}  1 & 0 \\  -2 & 4 \\ 3 & 5 \end{bmatrix} \begin{bmatrix} -2 & 3 \\ -4 & 1 \end{bmatrix} \\ \\
&= \begin{bmatrix} 1(-2)+0(-4) & 1(3)+0(1) \\ -2(-2)+4(-4) & -2(3)+4(1) \\ 3(-2)+5(-4) & 3(3)+5(1) \end{bmatrix}\\ \\
&= \begin{bmatrix} -2 & 3 \\ -12 & -2 \\ -26 & 14 \end{bmatrix}
\end{align*}
\\

In general, matrix multiplication is \underline{not} commutative, that is \(AB \neq BA\)\\ \\

Remark: System of linear equation and matrices\\
A system (\(\ast\)) of \(m\) linear equations in \(n\) variables can be reresented by matrix multiplication as follows:\\

\begin{align*} \begin{bmatrix} a_{11} & a_{12} & a_{13} & \ldots & a_{1n} \\ a_{21} & a_{22} & a_{32} & \ldots & a_{2n} \\ a_{31} & a_{32} & a_{33} & \ldots & a_{3n} \\ \vdots & \vdots & \vdots & \ddots & \vdots \\ a_{m1} & a_{m2} & a_{m3} & \ldots & a_{mn} \end{bmatrix} \begin{bmatrix} x_1 \\ x_2 \\ x_3 \\ \vdots \\ x_n \end{bmatrix} &= \begin{bmatrix} b_1 \\ b_2 \\ b_3 \\ \vdots \\ b_m \end{bmatrix} \\ \\ 
&\Leftrightarrow \, Ax = b
\end{align*}
\\
hence (\(\ast\)) \(\Leftrightarrow Ax = b\)\\

Example:\\
Consider the system
\begin{align*} (\ast) \quad \systeme{x-2y+3z=1,-x+4y+z=4, 2x-y=3} \end{align*}\\
(\(\ast\)) is a system of \(m=3\) linear equations in \(n=3\) variables\\ \\
Notice that (\(\ast\)) \(\Leftrightarrow \begin{bmatrix} 1 & -2 & 3 \\ -1 & 4 & 1 \\ 2 & -1 & 0 \end{bmatrix} \begin{bmatrix} x \\ y \\ z \end{bmatrix} = \begin{bmatrix} 1 \\ 4 \\ 3 \end{bmatrix}\)\\

Example:\\ \\
Let \(A = \begin{bmatrix} 1 & -2 & 1 \\ 2 & 3 & -1 \end{bmatrix}\)\\ \\

and \(x = \begin{bmatrix}  x_1 \\ x_2 \\ x_3 \end{bmatrix}\)\\ \\

and \(b = \begin{bmatrix} 0 \\ 0 \end{bmatrix} = 0\)
\\

Solve the matrix equation: \(Ax = b\)\\

We have \(Ax = b\)
\begin{align*} &\Leftrightarrow \begin{bmatrix} 1 & -2 & 1 \\  2 & 3 & -1 \end{bmatrix}\begin{bmatrix}  x_1 \\ x_2 \\ x_3 \end{bmatrix} = \begin{bmatrix} 0 \\ 0 \end{bmatrix}\\ \\
&\Leftrightarrow \begin{bmatrix} x_1 - 2x_2 + x_3 \\ 2x_1 +3x_2 - x_3 \end{bmatrix} \begin{bmatrix} 0 \\ 0 \end{bmatrix}\\ \\
&\Leftrightarrow \systeme{x_1-2x_2+x_3=0, 2x_1 + 3x_2 - x_3 = 0}
\end{align*}
\\

The corresonding augmented matrix is:
\\

\begin{align*} &\begin{bmatrix} 1 & -2 & 1 & 0 \\ 2 & 3 & -1 & 0 \end{bmatrix} \\ \\
&\begin{bmatrix} 1 & -2 & 1 & 0 \\ 0 & 7 & -3 & 0 \end{bmatrix}\\ \\
&\begin{bmatrix} 1 & -2 & 1 & 0 \\ 0 & 1 & -\frac{3}{7} & 0 \end{bmatrix} 
\end{align*}
\\

the the corresonding system of linear equations is:
\begin{align*} \systeme{x_1-2x_2+x_3=0, x_2 -\frac{3}{7}x_3 = 0} \quad \quad \begin{aligned} &\textcircled{1} \\ &\textcircled{2} \end{aligned} \end{align*}
\\

Let \(x_3 = t\), \(t \in \mathbb{R}\) is a parameter\\
then\\
\(\textcircled{2} \rightarrow x_2 = \frac{3}{7}x_3 = \frac{3}{7}t\)\\
and \(\textcircled{1} \rightarrow x_1 = 2x_2 - x_3 = 2(\frac{3}{7}t) - t = \frac{6}{7}t - t = -\frac{t}{7}\)
\\

then \(x = \begin{bmatrix} x_1 \\ x_2 \\ x_3 \end{bmatrix} = \begin{bmatrix} -\frac{1}{7}t  \\ \frac{3}{7}t \\  t \end{bmatrix} = t \begin{bmatrix} -\frac{1}{7} \\ \frac{3}{7} \\ 1 \end{bmatrix}\)\\

Then the solution set of (\(\ast\)) is\\

\begin{align*} \left\{ x=\begin{bmatrix} -\frac{1}{7} \\ \frac{3}{7} \\ 1 \end{bmatrix}, \, t \in \mathbb{R} \right\} \end{align*}\\
hence (\(\ast\)) has infinitely many solutions, and it is consistent.\\ \\

Partitioned matrices:\\
We can represent the system (\(\ast\)) or equivalently \(Ax=b\) by partitioning matrix \(A\) and \(x\) as follows:\\

\begin{align*} &\Leftrightarrow \begin{bmatrix} a_{11}x_1 & a_{12}x_2 & a_{13}x_3 & \ldots & a_{1n}x_n \\ a_{21}x_1 & a_{22}x_2 & a_{32}x_3 & \ldots & a_{2n}x_n \\ a_{31}x_1 & a_{32}x_2 & a_{33}x_2 & \ldots & a_{3n}x_n \\ \vdots & \vdots & \vdots & \ddots & \vdots \\ a_{m1}x_1 & a_{m2}x_2 & a_{m3}x_3 & \ldots & a_{mn}x_n \end{bmatrix} = b \\ \\
&\Leftrightarrow x_1 \begin{bmatrix} a_{11} \\ a_{21} \\ a_{31} \\ \vdots \\ a_{m1} \end{bmatrix} + x_2 \begin{bmatrix} a_{12} \\ a_{22} \\ a_{32} \\ \vdots \\ a_{m2} \end{bmatrix} + x_3 \begin{bmatrix} a_{13} \\ a_{23} \\ a_{32} \\ \vdots \\ a_{m3} \end{bmatrix} + x_n \begin{bmatrix} a_{1n} \\ a_{2n} \\ a_{3n} \\ \vdots \\ a_{mn} \end{bmatrix} = b \\ \\
\end{align*}
\(A_i\) is of size \(m*1\) for \(1 \leq i \leq n\)\\

\begin{align*} \Leftrightarrow x_1 A_1 + x_2 A_2 + x_3 A_3 + x_n A_n = b \end{align*}\\

here \(A_1, \, A_2, \, A_3, \ldots, \, A_n\) for a partition of matrix \(A\)\\
(i.e. \(A = \begin{bmatrix} A_1 & A_2 & A_3 & \ldots & A_n\end{bmatrix}\))\\

hence \(Ax = x_1 A_1 + x_2 A_2 + x_3 A_3 + \ldots + x_n A_n = b\) \(\quad\) (\(2\ast\))\\
This is a linear combination of the \(m*1\) matricies \(Ai\)'s (\(1 \leq i  \leq n\))
\\ \\
Remark:\\
System (\(\ast\)): \(Ax+b\) is consistent \(\iff\) b can be exressed as such a linear combination as given in (\(2\ast\)), where: \(x_1, \, x_2, \, x_3, \ldots, x_n\) are solutions of the system.























\end{document}