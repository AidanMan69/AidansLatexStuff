\documentclass{jhwhw}
\author{Aidan Garcia}
\usepackage[utf8]{inputenc}
\usepackage{mathrsfs}
\usepackage{graphicx}
\usepackage{amsmath}
\usepackage{amsfonts}
\usepackage{amssymb}
\usepackage{spalign}
\usepackage{nicefrac}
\usepackage{tabto}
\usepackage{parskip}
\title{1.1: An Introduction to Systems of Linear Equations}

\begin{document}

A linear equation in \(n>1\) variables \(x_1,x_2,x_3,\ldots,x_n\) has the form
\\
(\(\star\)) \(a_1 x_1 + a_2 x_2 + a_3 x_3 +\ldots+a_n x_n=b\)
\\

here \(a_i\) is a constant for any \(1 \leq i \leq n\) and \(b\) is a constant
\\

\(a_i\) are the coefficients of (\(\star\)) \\
\(x_i\) are the variables \\
\(a_1\) is the leading coefficient \\
\(x_1\) is the leading variable
\\

\textbf{Example: the following equations are linear}
\\
a) \(2x-3y=1\) (2 variables) \\
\(\rightarrow\) represents a line of the xy plane 
\\
b) \(x+y-z=2\) (3 variables) \\
\(\rightarrow\) represents a plane in the xyz coordinate system
\\ \\

\textbf{Example: the following equations are non-linear}
\\
1) \( \boxed{x^2} + y + z =0\) \(\leftarrow\) non-linear function \(x\)
\\
2) \(\boxed{xy} + z = -1\) \(\leftarrow\) x and y are not independent of eachother
\\
3) \(\boxed{\cos x} + y + z = 3\) \(\leftarrow\) non-linear function of \(x\)
\\ \\

\underline{Definition (Solution and Solution Set of (\(\star\))):}
\\

A solution of (\(\star\)) is a sequence of n real numbers  \(s_1, s_2, s_3,\ldots,s_n\) that satisfy  (\(\star\))
\\

The set of solutions of (\(\star\)) is called a solution set
\\

Remark: The solution of (\(\star\)) is a linear equation in \(n\) variables does have a parametric representation.
\\ \\

\textbf{Example: Find a parametric representation of each linear equation}
\\

a) \(\boxed{1}\) \(2x_1 - 3x_2 = 1\) (a linear equation in \(n=2\) represents a line in \(\mathbb{R} ^2\))
\\
let \(x_2=t\), here \(t \in \mathbb{R}\)
\\

\begin{align*} 
\text{then } \boxed{1} &\Leftrightarrow 2x_1 - 3t = 1 \\
&\Leftrightarrow 2x_1 = 1 + 3t \\
&\Leftrightarrow x_1 = \frac{1}{2} + \frac{3}{2}t \\
\end{align*}

Then the solution set of \(\boxed{1}\) is 
\\
\(\left\{(x_1 , x_2) | x_1 = \frac{1}{2} + \frac{3}{2}t \text{ and } x_2 = t \text{, } t \in \mathbb{R} \right\}\)
\\
\(\boxed{= \left\{\left(\frac{1}{2} + \frac{3}{2}t, t\right) | t \in \mathbb{R} \right\}}\) A paramaterization of the solution set
\\ \\

b) \(\boxed{2}\) \(x_1 + 2x_2 - x_3 = 4\)
\\
Let \(x_3 = t\), \(t \in \mathbb{R}\) is a parameter
\\
Let \(x_2 = s\), \(s \in \mathbb{R}\) is a parameter
\\

then \(\boxed{2}\) \(x_1 + 2s - t = 4\) \\
\(\Leftrightarrow x_2 = -2s + t + 4\), \(t \in \mathbb{R}\) and \(s \in \mathbb{R}\)
\\ \\
Then the solution set of \(\boxed{2}\) is \\
\(\left\{(x_1, x_2, x_3) | x_1 = -2s + t + 4, x2 = s, x_3 = t, s \in \mathbb{R} \text{ and } t \in \mathbb{R} \right\}\)
\\
\(\boxed{= \left\{(-2s + t + 4, s, t) | s \in \mathbb{R} \text{ and } t \in \mathbb{R} \right\}}\)
\\ \\

\underline{Definition (System of linear equations in \(n \geq 1\) variables):}
\\

A system of \(m \geq 1\) linear equation in \(n \geq 1\) variables is a set of \(m\) equations, each of which is linear in the same \(n\) variables
\\

(\(\star\)) \begin{align*} 
a_{11} x_1 + a_{12} x_2 + a_{13} x_3 + \ldots + a_{1n} x_n &= b_1\\
a_{21} x_1 + a_{22} x_2 + a_{23} x_3 + \ldots + a_{2n} x_n &= b_2\\
a_{31} x_1 + a_{32} x_2 + a_{33} x_3 + \ldots + a_{3n} x_n &= b_3\\
\vdots\\
a_{m1} x_1 + a_{m2} x_2 + a_{m3} x_3 + \ldots + a_{mn} x_n &= b_m
\end{align*}
\\

A solution of (\(\star\)) is a sequence of numbers \(s_1, s_2, s_3, \ldots, s_n\) that is a solution of each equation of the system.
\\

Remark (On the number of solutions of a system (\(\star\)) of linear equations):
\\

For a given system of linear equations (\(\star\)) Precisely one of the following statements is true:
\\

1) The system (\(\star\)) has exactly one solution (system is consistent) \\
2) The system (\(\star\)) has infinitely many solutions (system is consistent) \\
3) The system (\(\star\)) has no solutions (system is inconsistent) \\










\end{document}
